\chapter{Implementation}
\label{sec:implementation}

This chapter describes the implementation of the Fuzzy Tuning technique in AutoPas. The implementation is divided into three main parts: the generic Fuzzy Tuning framework, the Tuning Strategy, and the Rule Parser. The Fuzzy Tuning framework is the core of this implementation and implements the mathematical foundation of this technique. The Tuning Strategy is the interface between the Fuzzy Tuning framework and the AutoPas simulation. It is responsible for interacting with AutoPas and updating the queue of configurations. The Rule Parser is responsible for parsing the rule base supplied by the user and converting it into the internal representation used by the Fuzzy Tuning framework.
The implementation of the Fuzzy Tuning technique in AutoPas is designed to be as generic as possible to allow for easy integration of new types of rule bases.

\section{Fuzzy Tuning Framework}

The Fuzzy Tuning framework implements the mathematical foundation of the Fuzzy Tuning technique. It consists of several components that work together to apply the Fuzzy Rules to the input variables and generate the output variables. The components of the Fuzzy Tuning framework are as follows:

\begin{itemize}
    \item \textbf{Crisp Set:} The Crisp Set is used to model k-cells used as the underlying sets over which the Fuzzy Sets are defined. A k-cell is a hyperrectangle in the k-dimensional space constructed from the Cartesian product of k intervals $I = I_1 \times I_2 \times \ldots \times I_k$ where $I_i = [x_{low}, x_{high}] \subset \mathbb{R} $ is an interval in the real numbers. They are used to define the underlying sets over which the corresponding fuzzy set is defined and can be thought of as the parameter space of the input variable.


    \item \textbf{Fuzzy Set:} A fuzzy set consists of a membership function that assigns a degree of membership to each element of the Crisp Set. There are two types of membership functions: The \texttt{BaseMembershipFunction} and the \texttt{CompositeMembershipFunction}. The \texttt{BaseMembershipFunction} implement a function $f: \mathbb{R} \rightarrow [0, 1]$ that directly maps the crisp value to the degree of membership. The \texttt{CompositeMembershipFunction} is used to create new fuzzy sets by recursively combining existing fuzzy sets and their membership functions with generic functions. This helps to split up the complex fuzzy sets of the rule base into smaller, more manageable parts. Lets say we have a fuzzy set $\tilde{A}$ defined over the Crisp Set $X$ and a fuzzy set $\tilde{B}$ defined over the Crisp Set $Y$. Both membership functions are functions mapping the respective Crisp Set to the degree of membership ($\mu_{\tilde{A}}: X \rightarrow [0, 1]$ and $\mu_{\tilde{B}}: Y \rightarrow [0, 1]$). Since both membership functions directly map the Crisp Set to the degree of membership, they are considered \texttt{BaseMembershipFunction}s. When we want to create a new fuzzy set $\tilde{C} = \tilde{A} \cap \tilde{B}$ this new fuzzy set is defined over the Crisp Set $X \times Y$  and thus needs to provide a membership function $\mu_{\tilde{C}}: X \times Y \rightarrow [0, 1]$. As described in \todo{add chcpter} this membership function is defined as $\mu_{\tilde{C}}(x, y) = \min(\mu_{\tilde{A}}(x), \mu_{\tilde{B}}(y))$, which can be thought of as recursively combining the membership functions of $\tilde{A}$ and $\tilde{B}$ with the minimum function.

          This way of combining fuzzy sets builds a tree structure where the leafs calculate a direct membership value and the inner nodes combine the membership values of their children and pass them up to their parent.

          \todo{add image}

    \item \textbf{Linguistic Variable:} A linguistic variable is a variable whose values are terms in a natural language. Each term is associated with a fuzzy set that defines its concept using the language of fuzzy logic. The linguistic variables can then be used to express rules in a human-readable way.

    \item \textbf{Fuzzy Rule:} A fuzzy rule is a conditional statement that describes the relationship between input- and output variables. It consists of an antecedent and a consequent both of which are fuzzy sets. During the evaluation of the rule, the antecedent is evaluated to determine the degree to which the rule is satisfied and thus the effect of the rule can be reduced accordingly.

    \item \textbf{Fuzzy System:} The Fuzzy System is the main component of the Fuzzy Tuning framework and collects all the rules and variables. It is responsible for evaluating the rules and passing the resulting defuzzified values back to the Tuning Strategy.
\end{itemize}

\todo {Add UML Diagram}

\section{Tuning Strategy}

\subsection*{Suitability Approach}

\subsection*{Individual Tuning Approach}