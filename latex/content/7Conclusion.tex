\chapter{Conclusion}
\label{sec:conclusion}

This thesis introduced a novel fuzzy logic-based tuning strategy for AutoPas and implemented a generic Fuzzy Tuning Library into the AutoPas framework, providing a reusable foundation for future research projects.

\medskip

\noindent A key contribution was the development of a data-driven approach to automatically generate competitive fuzzy systems, enabling the tuning of complex systems without extensive prior knowledge. The results demonstrated that the proposed Fuzzy Tuning Strategy significantly outperformed existing tuning strategies on selected benchmarks, substantially reducing the total simulation runtime by up to a factor of 1.96.

\medskip

\noindent While the fuzzy tuning strategy and the data-driven rule generation process showed promise, they are not universal solutions, as considerable upfront effort is necessary to collect the data to generate the rule base. Since such a workload cannot be expected from typical users of AutoPas, future research is needed to streamline the data collection and rule generation process to make the fuzzy tuning strategy more accessible to a broader audience.

\medskip

\noindent In conclusion, the Fuzzy Tuning Strategy and the data-driven rule generation process represent a significant step forward in tuning AutoPas simulations and offer a solid foundation for future research. While challenges remain in making it more accessible and broadly applicable, the potential for substantial performance gains makes this an exciting area for continued investigation and refinement.