%%% SETTINGS

% no word wrapping
%\righthyphenmin=62
%\lefthyphenmin=62
% fewer hyphens
\usepackage{microtype}

% german symbols
\usepackage[utf8]{inputenc}

% strikethrough by \sout
\usepackage[normalem]{ulem}

% insert graphics
\usepackage{graphicx}
% more flexible figures e.g. graphics with captions beside them
\usepackage{floatrow}
% more flexible captions.
% Use \captionsetup{options} to configure,
% use it in an environment for local setup
\usepackage{caption}
% subfigures (see template):
\usepackage{subcaption}

% more control of enumerations and itemizations
\usepackage{enumitem}
% less space between items
\setlist[itemize]{itemsep=0cm}
\setlist[enumerate]{itemsep=0cm}
% more customizeable tables (e.g. multiple lines per cell)
\usepackage{tabularx}
% fix for vertical centering
\usepackage{ragged2e}
\renewcommand\tabularxcolumn[1]{>{\Centering}m{#1}}
% column types with multiple lines and formatting
\usepackage{array}
\newcolumntype{C}{>{\centering\arraybackslash}X}
\newcolumntype{R}{>{\raggedleft\arraybackslash}X}
\newcolumntype{L}{>{\raggedright\arraybackslash}X}
% merge multiple rows \multirow{2}{*}{bla} & \\ &
\usepackage{multirow}
% activate for tables with page breaking
%\usepackage{ltablex}
% fix for table movement and itemizations
%\keepXColumns

% fix for dynamics spaces after custom commands
\usepackage{xspace}

% tabbing: use with \tab
\usepackage{tabto}
\TabPositions{4cm}

%% fancy math
% propper matrices, underbrace text
%\usepackage{amsmath}
\usepackage{mathtools}
% special symbols e.g. squares
\usepackage{amssymb}

%% plotting
\usepackage{pgfplots}
\pgfplotsset{compat=1.18}
\usepgfplotslibrary{fillbetween}

%%Settings for code
% code placement right there
\usepackage{float}
% code coloring
\usepackage{xcolor}
% code listing
\usepackage{listings}
\usepackage{scrhack}

% flexible multi column style
\usepackage{multicol}

% graphs
\usepackage{tikz}
\usetikzlibrary{shapes.geometric, arrows, calc, shapes.arrows, arrows.meta, bending, backgrounds}
% define some elements
\tikzstyle{startstop} = [rectangle, rounded corners, minimum width=3cm, minimum height=1cm,text centered, draw=black, fill=blue!30]
\tikzstyle{arrow} = [thick,->,>=stealth]

\usepackage{pgfmath} % for calculations in tikz
\usepackage{graphicx} % for images

% Some code highlighting styles you can use with lstlistings
% C++ code style similar to default eclipse
\lstdefinestyle{eclipse-cpp} {
    captionpos=b,
    language=C++,
    otherkeywords={final},
    basicstyle=\footnotesize,
    numbers=left,
    numberstyle=\small,
    showstringspaces=false,
    tabsize=2,
    frame=single,
    breaklines=true,
    keywordstyle=\bfseries\color[RGB]{127,0,85},
    identifierstyle=\color[RGB]{0,0,192},
    stringstyle=\color[RGB]{42,0,255},
    commentstyle=\color[RGB]{63,127,95},
}

% If no highlighting is intended
\lstdefinestyle{plain}{
}

% fancy algorithms (see template)
\usepackage[ruled, vlined, linesnumbered]{algorithm2e}
\DontPrintSemicolon
\SetKw{KwBy}{by}
\SetKw{KwAnd}{and}

% clickable links and clickable table of content <3
% Options: links with linebreaks
\PassOptionsToPackage{hyphens}{url}\usepackage[bookmarks=false]{hyperref}
\hypersetup{
    colorlinks,
    citecolor=black,
    filecolor=black,
    linkcolor=black,
    urlcolor=black
}
% Alterations to labels used by \autoref{}: Capitalize everyything
\def\chapterautorefname{Chapter}
\def\sectionautorefname{Section}
\def\subsectionautorefname{Subsection}
\def\algorithmautorefname{Algorithm}
\def\subfigureautorefname{Figure}
% for fully custon stuff use:
% \hyperref[custom:foo]{Custom~\ref*{custom:foo}}


\usepackage{bookmark} % for better bookmarks

\usepackage[section,numberedsection=autolabel]{glossaries}
\usepackage[automake]{glossaries-extra} % for glossaries
\makeglossaries

\usepackage{enumitem}


\definecolor{codegreen}{rgb}{0,0.6,0}
\definecolor{codepurple}{rgb}{0.8,0.2,0.22}
\lstdefinelanguage{FuzzyLanguage}{
    language=[Sharp]C,
    basicstyle=\tiny\ttfamily,
    keywordstyle=\color{blue},
    keywordstyle = [2]{\color{orange}},
    commentstyle=\color{codegreen},
    stringstyle=\color{red},
    frame=single,
    captionpos=b,
    breaklines=true,
    breakatwhitespace=false,
    morekeywords=     {if, then, FuzzyVariable,OutputMapping, FuzzySystemSettings  },
    morekeywords = [2]{Gaussian, SigmoidFinite},
    stringstyle=\color{codepurple},
    showstringspaces=false,
    aboveskip=0.5em,
    belowskip=0.5em,
}


\pgfmathdeclarefunction{step}{3}{%
    \pgfmathparse{#1 * #3<= #2 ? 0 : 1}%
}

\pgfmathdeclarefunction{sigmoid}{3}{%
    \pgfmathparse{1/(1 + exp(-(#1-#2)*#3))}%
}

\pgfmathdeclarefunction{gaussian}{3}{%
    \pgfmathparse{exp(-((#1-#2)^2)/(2*#3^2))}%
}

\pgfmathdeclarefunction{trapezoid}{5}{%
    \pgfmathparse{max(0, min((#1-#2)/(#3-#2), 1, (#5-#1)/(#5-#4)))}%
}

\newcommand*\calc[1]{%
    \pgfmathparse{#1}%
    \pgfmathprintnumber{\pgfmathresult}%
}

\newcommand\YAMLcolonstyle{\color{red}\mdseries}
\newcommand\YAMLkeystyle{\color{black}\bfseries}
\newcommand\YAMLvaluestyle{\color{blue}\mdseries}


\newcommand\language@yaml{yaml}

\expandafter\expandafter\expandafter\lstdefinelanguage
\expandafter{\language@yaml}
{
keywords={true,false,null,y,n},
keywordstyle=\color{darkgray}\bfseries,
basicstyle=\YAMLkeystyle,                                 % assuming a key comes first
sensitive=false,
comment=[l]{\#},
morecomment=[s]{/*}{*/},
commentstyle=\color{purple}\ttfamily,
stringstyle=\YAMLvaluestyle\ttfamily,
moredelim=[l][\color{orange}]{\&},
moredelim=[l][\color{magenta}]{*},
moredelim=**[il][\YAMLcolonstyle{:}\YAMLvaluestyle]{:},   % switch to value style at :
morestring=[b]',
morestring=[b]",
literate =    {---}{{\ProcessThreeDashes}}3
{>}{{\textcolor{red}\textgreater}}1
{|}{{\textcolor{red}\textbar}}1
{\ -\ }{{\mdseries\ -\ }}3,
}

% switch to key style at EOL
\lst@AddToHook{EveryLine}{\ifx\lst@language\language@yaml\YAMLkeystyle\fi}

\newcommand\ProcessThreeDashes{\llap{\color{cyan}\mdseries-{-}-}}